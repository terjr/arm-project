\begin{abstract}

In order to build more energy efficient traditional hardware, we are in need for
better understanding of how the ISA and its implementation relates to energy
efficiency. This paper investigates the instruction level energy efficiency of
an ARM Cortex-A9, and presents numbers on current drain, pipeline utilization
and other performance data for different instructions. This allows reasoning about
how instructions are executed and how energy efficient their implementation are
in this otherwise publicly closed architecture.

The testbench used in the experiments measures the core current drain by
utilizing a shunt resistor connected directly between the CPU core voltage
supply pins and an external power supply. Measuring isolated core current drain
effectively means skipping other on- and off-chip peripherals as much as possible,
and get a more decent view of what is going on within the processor core. We
also utilize a small and very core-local instruction cache unit called fast-loop-mode,
this together with not using data from memory leaves the caches totally unused.

The results shows that the ARM ISA is well balanced regarding commonly used
instructions, with instructions such as \texttt{add} and \texttt{sub} seemingly
most energy efficient. Unfortunately the ISA suffers from an inefficient handling
of conditional execution and status flags. Both terms cause what is assumed to
be forced synchronization, which again forces inefficient utilization of the
otherwise computationally strong core.

\end{abstract}
