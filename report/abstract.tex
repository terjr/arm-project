\begin{abstract}

Modern microprocessors are limited by power density and new designs must
emphasize energy efficiency to become successful. Building energy efficient
hardware requires better understanding of how the ISA and its implementation
relates to energy efficiency. This paper investigates the instruction level
energy efficiency of an ARM Cortex-A9 and presents current drain and
pipeline utilization for different instructions. These numbers reveal
information about how instructions are executed and how energy efficient their
implementations are in this proprietary architecture.

The testbench used in the experiments relies on accurate energy measurements.
This is achieved by measuring voltage drop over a shunt resistor placed between
the CPU core voltage supply pins and an external power supply. Memory usage is
avoided by exploiting instruction buffers and omitting loads/stores. This
renders the memory hierarchy unused, isolating the core as much as possible.

The results shows that the ARM ISA is well balanced. Commonly used
instructions such as \texttt{add}, \texttt{sub} and \texttt{mul}
seems to be energy efficient. Unfortunately, the ISA suffers from an
inefficient handling of conditional execution and status flag updates. Such
instructions seems to force synchronization, which then leads to
inefficient utilization of the otherwise computationally strong core.

\end{abstract}
