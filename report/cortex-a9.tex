\section{Description of the ARM Cortex A9 CPU}

% TODO: Terje wants this section to be under Methology

The processor we used in the experiment was an ARM Cortex A9
r3p0\footnote{Revision number revealed by printing the contents of the Main ID
Register}, hereby denoted A9. This processor runs at a static frequency of
1.7GHz, have 4 32-bit cores, each with its own out-of-order dual issue
speculative pipelines\cite{armtech}. The pipeline is split after the dispatch
stage into 4 different lines, each with its own functional units. The different
pipelines is not very well documented, but our experiments together with the
common subset of information found in various
documentations\cite{armtech}\cite{7cpu}\cite{lotofdocs}, the four pipelines are
structured as follow: Main execution pipeline with a  general ALU, and a
hardware-multiply, secondary execution pipeline with only a general ALU,
load-store-pipeline, containing only an hardware adder to generate addresses,
and a floating-point pipeline containing an ordinary FPU and connections to the
NEON unit.

It is a common philosophy that a RISC should consume one clock cycle pr.
instruction\cite{unknown}.  For this particular processor made by Advanced RISC
Machines, it is not the case. The dual issue and its parallell general ALU
pipelines enables the processor to achive more than one instruction pr. clock
cycle, and at the same time, it supports multiply, divide and floating point,
each taking multiple clock cycles. It is even so that multiply, divide and
floating point takes different amount of cycles according to their surrounding
instructions.

The A9 processor contains an Performance Monitor Unit with 6 generic event
counters, a cycle counter and 58 different events that are mappable to the event
counters\cite{armtech}. A list of possible events can be found in table A.18 in
the Cortex-A9 Technical Reference Manual\cite{armtech}.


