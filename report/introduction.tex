\section{Introduction}

\IEEEPARstart{M}{aking} processors burn less energy while at the same time
increasing performance is currently one of the greatest challenges hardware
designers are facing. Performance alone can be improved by cramming more
components onto integrated circuits \cite{moore1965cramming} utilizing new
process technologies. However, due to the end of Dennard scaling \cite{frank2001device},
power density on chip will increase linearly with transistor count. Computer
designers are forced to employ novel techniques mitigating this issue, such as
shutting down parts of the chip \cite{esmaeilzadeh2011dark} and designing
specialized hardware.

As processors grow more sophisticated, it becomes harder to reason about their
energy efficiency. Even RISC processors, which traditionally were designed to
be simple\cite{sivarama}, have seen a steep increase in complexity during the
last decade\cite{alf_egil_bogen_cisc_risc_blog}. Features that previously only
existed in CISC processors are now entering the RISC domain; more complex
operations are done per clock cycle. Current RISC designs may have deep
pipelines, increased component complexity, advanced branch predictor units and a
high degree of instruction level parallelism. They include features that aims to
reduce energy consumption and increase throughput in return for added
complexity. Moreover, processors are increasingly designed to integrate
seamlessly with external components such as accelerators and the memory system.

The need for energy efficient processors is increasing. To better understand how
execution of different instructions contribute to energy consumption we propose
a method to measure energy efficiency at the instruction level of a processor.
Being able to model and monitor energy consumption has recently got the
industry's attention: Semiconductor companies are making software and hardware
targeting the embedded market providing a monitor for energy consumption
on-chip, giving application engineers the opportunity to optimize for energy.

With the emergence of performance counters, it is now possible to detect how
different pipeline stages affects the overall power consumption. Previous
research correlates power drain seen from the wall outlet with performance
counters on the CPU\cite{singh,bertran,bircher}. Others look at
energy usage under different workloads\cite{carroll2010analysis}.

In this paper, we analyze the instruction level energy efficiency of a modern
RISC architecture. We isolate as many architecture components as possible
by correlating performance counters with observed current drain for specific
instructions. We also show that it is feasible to get a per-instruction energy
overview of an existing architecture by understanding the hardware and writing
benchmark programs. We look at simple single-cycle instructions as well as the
most complex instructions using multiple cycles on our target processor. When
each relevant instruction has been measured, we compare their normalized energy
consumption and discuss properties of different instructions.

This work is motivated by and related to the SHMAC
project\cite{ntnushmac,Umuroglu662354,rusten2012implementing} at NTNU. The goal
of the SHMAC project is to build an energy efficient heterogeneous many-core
computer architecture: multiple processing cores with different capabilities
share a single ISA and can be put on the same die to create a processor tailored
for a specific application. Exploring instruction level energy efficiency gives
us a better view on how different ISA implementations perform at different
tasks\cite{kumar2003single}. This information can be useful in the design phase
of novel computer architectures such as SHMAC.

Another use for this kind of information is that compilers can optimize code for
energy efficiency and not only performance. Further, it can enable simulators to
estimate energy consumption of a given program and even compare energy
efficiency on different cores, as shown to be effective in \cite{kumar2003single}.

\vfill
