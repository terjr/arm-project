\section{Introduction}

\IEEEPARstart{M}{aking} processors burn less energy while at the same time
increasing performance is currently one of the greatest challenges hardware
designers are facing. Performance alone can be improved by cramming more
components onto integrated circuits \cite{moore1965cramming} by utilizing new
processing technologies. However, it also incurs additional generation of heat
which the surroundings must manage to dissipate. Thus, hardware designers has
been forced to come up with new techniques, such as shutting down parts of the
chip \ref{dark_silicon} and designing specialized hardware.

Traditionally, RISC architectures has been influenced by design choices making
each instruction as simple as possible\cite{sivarama}. E.g. the AVR architecture
designed in the early 90's has a two-stage pipeline and only a few instructions
consumes more than one cycle. More recently, however, even RISC architectures
includes features that aims to reduce energy consumption and increase throughput
in return for nondeterminism and added complexity. Multiple pipelines, dual
issuing and parallel general ALUs are entering the RISC domain and is soon to
become ubiquitous even in the embedded world.

As processors grow more sophisticated, it becomes harder to reason about their
energy efficiency. Pipeline components are getting more complex, with advanced
branch predictior units and a high degree of instruction level parallelism. The
processor core itself gets more integrated with the rest of the system and many
parts of the pipeline contribute to the overall power consumption\cite{bertran}.
Thus, we need to gain insight of the processors energy characteristics by
running real-world measurements on a specific processor. Compilers can then use
these results to optimize for energy and not only performance.


\todo[inline]{Write about previous work here, but do not dedicate a subsection
    to it.}
Previous work in this field has been concentrated around measuring power drain
from the wall outlet, and then correlating the consumed energy with performance
counters available on the CPU\cite{singh}\cite{bertran}\cite{bircher}. Others
have done similar test setup, but with a focus workload classes instead of
instruction level energy consumption\cite{carroll2010analysis}.

Being able to model and monitor energy consumption has also got the industry's
attention. Semiconductor companies are making software targeting the embedded
market to monitor energy consumption on-chip, giving application engineers the
opportunity to optimize for energy.



\subsection{Contributions}
This paper investigates the instruction set architecture and how its properties
relates to energy consumption. In this paper, this investigation is realized on
an ARM Cortex A9.

In this paper, we present results after analyzing the instruction level energy
efficiency of a modern computer architecture. We take a set of commonly used
microarchitecture features, like out-of-order execution and dual-issue, into
account in order to isolate the functional unit's energy consumption.


Our research is highly motivated by the SHMAC project from the Energy Efficient
Computing Systems research group at NTNU. SHMAC is The Single-ISA Heterogeneous
Many-core Computer where one of the main goals is an optimal processor in terms
of energy efficiency. The SHMAC processor is based on the Amber open-source ARM
compatible processor core project. Our research on the ARM Cortex A9
based Exynos 4212 will of course not be equal to the same experiments run on
a final implementation of SHMAC, but it will give concrete results for a near
ISA-equivalent chip.

