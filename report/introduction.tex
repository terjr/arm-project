\section{Introduction}

\IEEEPARstart{M}{aking} processors burn less energy while at the same time
increasing performance is currently one of the greatest challenges hardware
designers are facing. Performance alone can be improved by cramming more
components onto integrated circuits \cite{moore1965cramming} by utilizing new
processing technologies. However, it also incurs additional generation of heat
which the surroundings must manage to dissipate. Thus, hardware designers has
been forced to come up with new techniques, such as shutting down parts of the
chip \cite{esmaeilzadeh2011dark} and designing specialized hardware.

As processors grow more sophisticated, it becomes harder to reason about their
energy efficiency. Even RISC processors, which is designed with simplicity in
mind\cite{sivarama}, has got a steep increase in complexity over the last
years\cite{alf_egil_bogen_cisc_risc_blog}. Features that previously only existed
in CISC processors are now entering the RISC domain. Current RISC designs may
have deep pipelines, increased component complexity, advanced branch predictor
units and a high degree of instruction level parallelism. They include features
that aims to reduce energy consumption and increase throughput in return for
nondeterminism and added complexity. Moreover, processors are designed as more
integrated components -- interacting more widely with the rest of the system.

Since energy efficiency are becoming more and more important at all levels of
computing, we need a broader understanding of how energy are consumed. Previous
research correlates power drain seen from the wall outlet with performance
counters on the CPU\cite{singh}\cite{bertran}\cite{bircher}. Others look at
energy usage under different workloads\cite{carroll2010analysis}. The SHMAC
project\cite{Umuroglu662354}\cite{rusten2012implementing} looks into energy
efficient hardware through a tile-based architecture scheme.

Being able to model and monitor energy consumption has also
got the industry's attention: Semiconductor companies are making software
targeting the embedded market to monitor energy consumption on-chip, giving
application engineers the opportunity to optimize for energy. With the emergence
of performance counters, it is now possible to detect how different pipeline
stages affects the overall power consumption\cite{bertran}.

In this paper we make two contributions. First, we analyze the instruction level
energy efficiency of a modern RISC computer architecture. We try to isolate as
many architecture components as possible by correlating performance counters
with observed power drain, emphasizing components that directly relates to the
pipeline activity for a specific instruction. Secondly, we show that it is
feasible to get a per-instruction energy overview of an existing architecture by
applying our method and writing benchmark programs. We look at both simple
single-cycle instructions as well as the most complex instructions consuming
multiple cycles on our target processor.

When each relevant instruction has been measured, we can compare their
normalized energy consumption and possibly identify patterns where consumption
is greater than expected. This can help in later architectural designs where
energy consumption is important. Also, with energy profiles on the instruction
level, compilers will be able to optimize their programs for energy efficiency.

The SHMAC project shows that it is possible to use the instruction energy
profiles to design hardware that is more energy efficient for a given task in
more sophisticated ways than previously possible. Energy profiles gives us a
better view of how different cores perform at different
tasks\cite{kumar2003single}. A system can then be composed of tiles with energy
profiles matching the given application.


%Our research is highly motivated by the SHMAC project from the Energy Efficient
%Computing Systems research group at NTNU. SHMAC is The Single-ISA Heterogeneous
%Many-core Computer where one of the main goals is an optimal processor in terms
%of energy efficiency. The SHMAC processor is based on the Amber open-source ARM
%compatible processor core project. Our research on the ARM Cortex A9
%based Exynos 4212 will of course not be equal to the same experiments run on
%a final implementation of SHMAC, but it will give concrete results for a near
%ISA-equivalent chip.

