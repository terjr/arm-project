\section{Introduction}

\IEEEPARstart{T}{odays} processors grow more sophisticated each day, and more
and more advanced features are added. In the earlier days, the focus was all on
performance, but current research are hitting walls where we cannot dispatch more heat.
Along with computers entering smaller and more energy aware domains, power and
energy has become a major focal point\cite{patterson}\cite{hennessy}.

This paper investigates the instruction set architecture and how its properties
relates to energy consumption. In this paper, this investigation is realized on
an ARM Cortex A9.

It is a common philosophy that a RISC should consume one clock cycle per
instruction \cite{sivarama}. This would make it a rather easy job to measure
and calculate the power consumption on any RISC processor. Even so, this is not
necessarily true any more.  Dual pipelines and dual issue and parallel general
ALUs are entering the RISC domain, allowing the processor to retire more than
one instruction per clock cycle. At the same time, processors that support
multiply, divide and floating point often use different amount of clock cycles
to perform these operations. It might even be so that multiply, divide and
floating point takes different amount of cycles according to their surrounding
instructions.

Our research is highly motivated by the SHMAC project from the Energy Efficient
Computing Systems research group at NTNU. SHMAC is The Single-ISA Heterogeneous
MAny-core Computer where one of the main goals is an optimal processor in terms
of energy efficiency. The SHMAC processor is based on the Amber open-source ARM
compatible processor core project. Our research on the ARM Cortex A9
based Exynos 4212 will of course not be equal to the same experiments run on
a final implementation of SHMAC, but it will give concrete results for a near
ISA-equivalent chip.

Previous work in this field has been concentrated around measuring power drain
from the wall outlet, and then correlating the consumed energy with performance
counters available on the CPU\cite{singh}\cite{bertran}\cite{bircher}. Others
have done similar test setup, but with a focus workload classes instead of
instruction level energy consumption\cite{carroll2010analysis}.

In this paper, we present results after analysing the instruction level energy
efficiency of a modern computer architecture. We take a set of commonly used
microarchitecture features, like out-of-order execution and dual-issue, into
account in order to isolate the functional unit's energy consumption.

\todo[inline]{What exactly have we been focusing on? Reverse-engineered the ARM
    Cortex-A9. ARM does not even tell how many functional units there is in the
    A9 core. With these results, we have been able to normalize power
    consumption per instruction, eliminating energy usage caused by the memory
    hierarchy. We are still treating the A9 core as a black box.}
