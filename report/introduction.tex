\section{Introduction}

\IEEEPARstart{M}{aking} processors burn less energy while at the same time
increasing performance is currently one of the greatest challenges hardware
designers are facing. Performance alone can be improved by cramming more
components onto integrated circuits \cite{moore1965cramming} utilizing new
process technologies. However, due to the end of Dennard scaling \cite{frank2001device},
power density on a chip will remain constant. Computer designers are forced to
employ novel techniques mitigating this issue, such as shutting down parts of
the chip \cite{esmaeilzadeh2011dark} and designing specialized hardware.

As processors grow more sophisticated, it becomes harder to reason about their
energy efficiency. Even RISC processors, which are designed with simplicity in
mind\cite{sivarama}, have seen a steep increase in complexity over the last
couple of years\cite{alf_egil_bogen_cisc_risc_blog}. Features that previously
only existed in CISC processors are now entering the RISC domain; more complex
operations are done per clock cycle. Current RISC designs may have deep
pipelines, increased component complexity, advanced branch predictor units and a
high degree of instruction level parallelism. They include features that aims to
reduce energy consumption and increase throughput in return for added
complexity.  Moreover, processors are designed as more integrated components --
interacting extensively with the rest of the system.

The need for energy efficient processors is increasing, and in order to get a
better understanding of which operations that are expensive in terms of energy,
we propose a method to measure energy efficiency on the instruction level of a
processor. Being able to model and monitor energy consumption has recently got
the industry's attention: Semiconductor companies are making software targeting
the embedded market providing a monitor for energy consumption on-chip, giving
application engineers the opportunity to optimize for energy.

With the emergence of performance counters, it is now possible to detect how
different pipeline stages affects the overall power consumption. Previous
research correlates power drain seen from the wall outlet with performance
counters on the CPU\cite{singh}\cite{bertran}\cite{bircher}. Others look at
energy usage under different workloads\cite{carroll2010analysis}.

This work is highly related to and motivated by the SHMAC
project\cite{ntnushmac}\cite{Umuroglu662354}\cite{rusten2012implementing} at NTNU. The goal of
the SHMAC project is to build an energy efficient heterogenous many-core
computer architecture: multiple processing cores with different capabilities
share a single ISA and can be put on the same die to create a processor tailored
for a specific application. Energy profiles provides us a better view of how
different ISA implementations perform at different tasks\cite{kumar2003single}.
SHMAC can benefit from energy profiles to design different cores that are more
energy efficient for different task.  This paper investigates an ARM
architecture, since SHMAC is implementing the non-protected ARMv2 ISA.

In this paper we make two contributions. First, we analyze the instruction level
energy efficiency of a modern RISC architecture. We try to isolate as many
architecture components as possible by correlating performance counters with
observed power drain for a specific instruction. We also show that it is
feasible to get a per-instruction energy overview of an existing architecture by
understanding the hardware and writing benchmark programs. We look at simple
single-cycle instructions as well as the most complex instructions consuming
multiple cycles on our target processor.

When each relevant instruction has been measured, we can compare their
normalized energy consumption and possibly identify patterns where consumption
is greater than expected. This can help computer designers recognize energy
bottlenecks in previous designs and enhance upcoming designs. This is directly
related to how we can help the SHMAC project designing better hardware. Even
another use for energy profiles on the instruction level is that compilers can
optimize their output for energy efficiency and not only performance.
Furthermore, it will also enable simulators to estimate energy consumption of a
given program and even compare its energy consumption across different cores.
