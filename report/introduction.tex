\section{Introduction}

\IEEEPARstart{M}{aking} processors burn less energy while at the same time
increasing performance is currently one of the toughest challenges hardware
designers are facing. Performance alone can be improved by cramming more
components onto integrated circuits \cite{moore1965cramming} by utilizing new
processing technologies. However, it also incurs additional generation of heat
which the surroundings must manage to dissipate. Thus, hardware designers has
been forced to come up with new techniques, such as shutting down parts of the
chip \ref{dark_silicon} and designing specialized hardware.

Traditionally, RISC architectures has been influenced by design choices making
each instruction as simple as possible\cite{sivarama}. E.g. the AVR architecture
designed in the early 90's has a two-stage pipeline and only a few instructions
consumes more than one cycle. More recently, however, even RISC architectures
includes features that aims to reduce energy consumption and increase throughput
in return for nondeterminism and complexity. Multiple pipelines and dual issue
and parallel general ALUs are entering the RISC domain, allowing the processor
to retire more than one instruction per clock cycle.


\todo[inline]{Add something about the need for energy modeling techniques.
Ideally, we could just feed our results to an algorithm that schedules for
energy instead of performance. Processors are harder to reason about, so we need
to gather statistics about them in a running environment.}

Being able to model and monitor energy consumption has also got the industry's
attention. Semiconductor companies are making software targeting the embedded
market to monitor energy consumption on-chip, giving application engineers the
opportunity to optimize for energy.

%This would make it a rather easy job to measure
%and calculate the power consumption on any RISC processor. Even so, this is not
%necessarily true any more.   At the same time, processors that support
%multiply, divide and floating point often use different amount of clock cycles
%to perform these operations. It might even be so that multiply, divide and
%floating point takes different amount of cycles according to their surrounding
%instructions.

\todo[inline]{Write about previous work here, but do not dedicate a subsection
    to it.}
Previous work in this field has been concentrated around measuring power drain
from the wall outlet, and then correlating the consumed energy with performance
counters available on the CPU\cite{singh}\cite{bertran}\cite{bircher}. Others
have done similar test setup, but with a focus workload classes instead of
instruction level energy consumption\cite{carroll2010analysis}.


Our research is highly motivated by the SHMAC project from the Energy Efficient
Computing Systems research group at NTNU. SHMAC is The Single-ISA Heterogeneous
Many-core Computer where one of the main goals is an optimal processor in terms
of energy efficiency. The SHMAC processor is based on the Amber open-source ARM
compatible processor core project. Our research on the ARM Cortex A9
based Exynos 4212 will of course not be equal to the same experiments run on
a final implementation of SHMAC, but it will give concrete results for a near
ISA-equivalent chip.


\subsection{Contributions}
This paper investigates the instruction set architecture and how its properties
relates to energy consumption. In this paper, this investigation is realized on
an ARM Cortex A9.

In this paper, we present results after analyzing the instruction level energy
efficiency of a modern computer architecture. We take a set of commonly used
microarchitecture features, like out-of-order execution and dual-issue, into
account in order to isolate the functional unit's energy consumption.

\todo[inline]{What exactly have we been focusing on? Reverse-engineered the ARM
    Cortex-A9. ARM does not even tell how many functional units there is in the
    A9 core. With these results, we have been able to normalize power
    consumption per instruction, eliminating energy usage caused by the memory
    hierarchy. We are still treating the A9 core as a black box.}
