\section{Introduction}

\IEEEPARstart{T}{odays} processors grow more sophisticated each day, and
more and more advanced features are added. In the earlier days, the focus
was all on performance, but as we are hitting walls where we cannot dispatch
more heat, along with computers entering smaller and more energy aware domains,
power and energy has become a major focal point.

In this paper, we present a method for analysing the instruction level energy
efficency of a new computer architecture. We take commonly used
microarchitecture features, like out-of-order execution and dual-issue, into
account in order to isolate the functional unit's energy consumption.

In our case study, we apply our method to an ARM Cortex-A9, a mdern 32-bit
processor implementing the ARMv7 instruction set. We inspect the power
profile of various instructions and see that even though the ARM
Cortex-A9 is both a computationaly strong and energy efficient CPU, there are
room for energy improvements.


\todo[inline]{What exactly have we been focusing on? Reverse-engineered the ARM Cortex-A9. ARM
does not even tell how many functional units there is in the A9 core. With these
results, we have been able to normalize power consumption per instruction,
eliminating energy usage caused by the memory hierarchy.

We are still treating the A9 core as a black box.}
