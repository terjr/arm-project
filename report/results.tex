\section{Results}

%\begin{figure*}
%    \centering
%    \includegraphics[width=1.1\textwidth]{figures/graph_0123_base_lowpower_mul-0c6}
%    \label{fig:allmul}
%\end{figure*}
%
%\begin{figure*}
%    \centering
%    \includegraphics[width=0.7\textwidth]{figures/graph_0123_base_mul-0c7}
%    \label{fig:allmul}
%\end{figure*}
%
%
%
%\begin{figure*}
%    \centering
%    \subfloat[][Multiply]
%    {
%        \includegraphics[width=0.49\textwidth]{figures/graph_123_base_mul-0c5}
%        \label{fig:somemul}
%    }
%    \subfloat[][Single Cycle]
%    {
%        \includegraphics[width=0.49\textwidth]{figures/graph_1_base_arith_data_logic-0c5}
%        \label{fig:singlecycle}
%    }
%    \caption{test}
%\end{figure*}

\begin{figure*}[ht]
    \centering
    \includegraphics[width=\textwidth]{figures/graph_01_base_arith_saturate_data_pack_logic-0c6}
    \caption{Energy profile of single-cycle instructions, excluding multiply.}
    \label{fig:singlecycle}
\end{figure*}


\subsection{Introduction}
In this section we present data gathered from our experiments on the ARM
Cortex-A9.

We distinguish between single-cycle instructions and multi-cycle instructions
because they behave differently in and around the execution pipeline.
Instructions consuming only one cycle are fairly easy to reason about as there
is no need to normalize energy consumption with respect to the cycle count (i.e.
time). However, it is important to also recognize CPU capabilities such as dual
issuing that we have on our processor: All single-cycle ALU instructions execute
pairwise in parallel (one in each ALU), giving a peak performance of two
instructions per CPU cycle. On the other hand, multi-cycle instructions needs to
be carefully considered. Typically, multi-cycle instructions divide work which
can be done only in a subset of the available CPUs (e.g. one) over several
cycles, lowering the average current draw. They consume less energy per
time step, but also do less useful work.

For all these reasons, we partition the measured data in two data sets; one for
single-cycle instructions and one for multi-cycle instructions.

In the graphs, all bars colored green means that the tested instruction is a
single-cycle instruction, the light blue is two-cycles, and dark blue is
three-cycle instructions. The red bar is the baseline for power measurement.
This baseline is an alias for the least power-consuming instruction we could
find, which is the {\ttfamily setend}-instruction. This instruction sets the
endianness regarding memory operations to eiter big or little
endian.\cite{armcompilerref}.

The results presented in the gaphs are enumerated as Ampere
$\cdot$ cycles, which means that instructions using more time is normalized by
adding up their power drain with the time used in the pipeline. The results are
not converted into watts, joule or anything else that would be more convenient
in order to state a number on each instructions energy consumption. This is
because this paper investigates how each instruction differs from other
instructions in the same ISA, and thus we look at the power drain at the
processor core when a given instruction resides within it, multiplied by the
number of cycles the instruction consumes. 

During measurements, the core voltage was keept stable at
$1.3V\pm50mV$ during testing. The measurements where done with as full
pipelines as possible, avoiding hazards and instruction loading as much as
possible. This means that instructions that utilize more parts of the processor
will most likely be more energy consuming than those using only few components.
This is also shown with our {\ttfamily baseline}-measurement, as we assume that
the {\ttfamily setend}-instruction merly changes some status flags.



%Some instructions use variable amount of time. This section will contain
%information about how we normalize and compare energy consumption of
%these instructions. It will be difficult to compare single cycle instructions
%to the multi cycle ones, as the single cycle instructions is often the ones
%utilizing more than one ALU at a time. Also, the multi-cycle ones will most
%likely pipeline up very differently than the single cycle ones. We will try to
%draw a conclusion about the results, but it is important to note the differences
%in the execution path of these two categories.

\subsection{Single-Cycle Instructions}
On our target CPU, 70 of the 116\footnote{118 including conditionals} tested
instructions falls into this category (leaving 46 multi-cycle instructions).

\begin{figure}
    \centering
    \includegraphics[width=0.4\textwidth]{figures/graph_01_base_cond-0c6}
    \caption{Energy profile showing conditional execution.}
    \label{fig:cond}
\end{figure}

The results in \autoref{fig:singlecycle} shows that the ordinary single-cycle instructions
does not differ very much. Instructions like {\ttfamily rev} and {\ttfamily sel} are on the
top, which can be explained by looking on how these instructions move nearly all bits in
the operands around, while {\ttfamily cmp} and the different shift-instructios are most likely
moving fewer bits around. A more interesting result is how instructions bearing the {\ttfamily s}-flag
seems to have a lover consumption than their non-{\ttfamily s}-companion. It is hard to reason about
such, since we do not have access to the inner workings of this processor.

The results from the conditional-execution scheme brought by this ISA are also an interesting matter. We
can see from \autoref{fig:cond} how different versions of {\ttfamily add} compares. In the figured test,
{\ttfamily addne} is committed every time, while {\ttfamily addeq} never has its results committed. It is
interesting to see that even though {\ttfamily addeq} is never committed, it uses almost as much power as
the other {\ttfamily add}s. We can assume that with the addition of out-of-order scheduling, conditional
execution might be harder to implement.

\autoref{fig:singlecycle} shows that the {\ttfamily nop}-instruction has a rather low power consumption. This
is a bit missguiding, as the {\ttfamily nop}-instruction is actually a {\ttfamily mov r0,r0}-instruction, and thus
has both a read-after-write and a write-after-write hazard on it self. This makes the {\ttfamily nop}-instruction
serialize it self, and it is hard to fill the pipeline with this instruction. With this in mind, it makes sense
that {\ttfamily nop} works in this way, as it is often used to fill out clock cycles with non-destructive work. It
would not make sense to optimize the {\ttfamily nop} instruction, as it would merly fail to complete it's goal as
a space-and-time filler.

\begin{figure*}
    \centering
    \includegraphics[width=\textwidth]{figures/graph_0123_base_mul-0c6}
    \caption{Energy profile of multiply instructions.}
    \label{fig:allmul}
\end{figure*}

\begin{figure}
    \centering
    \includegraphics[width=0.48\textwidth]{figures/graph_023_base_quad_saturate_extend-0c6}
    \caption{Energy profile of multi-cycle instructions, excluding multiply.}
    \label{fig:multicycle}
\end{figure}


\subsection{Multi-Cycle Instructions}

See \autoref{fig:multicycle} and \autoref{fig:allmul}
\begin{itemize}
    \item The difference between instructions are much larger in the multi-cycle
        graphs.
    \item Talk about mul, and why it might differ that much (especially 2 and 3 cycle ones)
\end{itemize}

\subsection{Evaluation}
\begin{figure}
    \centering
    \includegraphics[width=0.48\textwidth]{figures/heat}
    \caption{Changes in heat and energy consumption for {\ttfamily add} at different runs together with heatsink and ambient temperature}
    \label{fig:heat}
\end{figure}

Each instruction was measured 41 times. For each run, there was small changes in
power consumption, but the relation trends was equal for every run.
As stated in \autoref{sec:temperature}, the power consumption is not easily pushed by temperature. \autoref{fig:heat} shows how the change in power consumption of the instruction
{\ttfamily add} over different runs combined with the ambient temperature the
heatsink temperature. According to these results, we assume that the change in
power consumption was not due to heat. We did not log heat for all test runs,
but assume that the results from \autoref{fig:heat} holds, and that this small
change in heat is not responsible for any disturbance in the power consumption
measurements.


