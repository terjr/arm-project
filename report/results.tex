\section{Results}

\begin{figure*}[ht]
    \centering
    \includegraphics[width=\textwidth]{figures/graph_01_base_arith_saturate_data_pack_logic-0c6}
    \caption{Energy profile of single-cycle instructions, excluding multiply}
    \label{fig:singlecycle}
\end{figure*}

\subsection{Introduction}
In this section we present data gathered from our experiments on the ARM
Cortex-A9. A brief description of each instruction can be found in the "ARM and
Thumb-2 Instruction Set Quick Reference Card"\cite{armasmref}. First, we discuss
the data from the performance counters. Together with the sparse official
documentation, this enables us to make some assumptions about how different
instructions are executed in the processor. Then we discuss the
results from the per instruction energy analysis.

\subsection{Decomposing the Core}
Instructions executed in the processor will utilize a subset
of all the available core components. By combining the components depicted in
\autoref{fig:pipeline} with the performance counter data listed in
\autoref{tab:perf_nonmul} and \autoref{tab:perf_mul}, we can deduce which
instructions that trigger what parts. We can also see how frequently each part
of the pipeline is used, as a fraction of cycle count and the given component
event counters.

\begin{table}
    \centering
    \begin{tabular}{|p{0.7cm}|R{0.8cm}R{0.6cm}R{0.85cm}R{0.6cm}R{0.45cm}R{0.75cm}R{0.72cm}|}
       \rowcolor{gray!50}
        \hline
        \centering
        Instr. &
        \centering
        Cycles &
        \centering
        Main Ex. &
        \centering
        Second Ex. &
        \centering
        Pred.&
        \centering
        Mis pred.&
        \centering
        No disp. &
        \begin{centering}
        Issue Empty
        \end{centering}
        \\
        \hline
        adc&1976&1762&1758&251&2&89&89\\
adcs&3599&1762&1756&251&2&1693&1690\\
add&1976&1762&1758&251&2&89&89\\
addeq&6596&1635&1883&251&2&4711&4711\\
addne&3350&1762&1758&251&2&1464&1464\\
adds&3598&1761&1757&251&2&1712&1712\\
and&1976&1762&1758&251&2&89&89\\
asr&1977&1762&1758&251&2&90&90\\
bic&1976&1762&1758&251&2&89&89\\
cpsid&14627&3516&1&251&2&11110&11110\\
eor&1978&1762&1758&251&2&91&91\\
lsl&1977&1762&1758&251&2&90&90\\
lsr&1978&1762&1758&251&2&91&91\\
mla&6608&6530&4895&251&2&76&76\\
mlas&15640&6529&12548&251&2&8858&74\\
nop&3642&3269&251&251&3&121&121\\
orr&1977&1762&1758&251&2&90&90\\
qadd&3600&1761&1757&251&2&1714&1714\\
qsub&3599&1761&1757&251&2&1713&1713\\
ror&1977&1762&1758&251&2&90&90\\
rrx&1976&1762&1758&251&2&89&89\\
rsb&1976&1762&1758&251&2&89&89\\
rsbs&3598&1761&1757&251&2&1712&1712\\
rsc&1976&1762&1758&251&2&89&89\\
sbcs&3599&1762&1756&251&2&1693&1690\\
setend&14627&3516&1&251&2&11110&11110\\
sub&1976&1762&1758&251&2&89&89\\
subs&3600&1761&1757&251&2&1714&1714\\
teq&3600&1762&1756&251&2&1694&1691\\
tst&3600&1762&1756&251&2&1694&1691\\

        \hline
    \end{tabular}
    \caption{Performance counter data from 252 iterations of all tested
    instructions, excluding multiply}
    \label{tab:perf_nonmul}
    \hfill
\end{table}


All results in \autoref{tab:perf_nonmul} and \autoref{tab:perf_mul} are gathered
by running each instruction included in our experiments using the template shown
in \autoref{list:inst_loop}. The cycle count (\emph{Cycles}) tells us how long
time, in terms of clock cycles, it took for the processor to execute the $252
\cdot (13+2) = 3780$ instructions. The loop has room for 13 test instructions,
while the last 2 is the loop head consisting of \texttt{subs} and \texttt{bne}.
\emph{Main Ex.} is the number of cycles where the main execution pipeline is
active, labeled ALU/MUL in \autoref{fig:pipeline}. \emph{Second Ex.} is for the
second execution pipeline, labeled ALU.  All instructions in our test bench have
a correct branch prediction count of 251 (\emph{Pred.}). This is most likely
because the first and the last iteration of the loop is mispredicted (\emph{Mis
pred.}). \emph{No disp.} is the number of cycles where there the processor was
unable to dispatch any instructions to any execution lane.
\emph{Issue Empty} is the number of cycles where there was no instructions in
the instruction queue. Note that we can see how many cycles the processor is stalling by looking at
the \emph{No disp.} and the \emph{Issue Empty} counters. When the \emph{No
disp.} number is higher than \emph{Issue Empty}, it means that the processor had
to stall due to hazards or intended flushing (e.g. \texttt{setend} flushes the
pipeline as all further issued instructions must follow the new endianess).
In special cases, such as our baseline instruction \texttt{setend}, we see
that the amount of \emph{No disp.} is very high, which again means that
the CPU is mostly stalling. It seems to be a strong relation between low
power usage and high stall numbers.

%Some instructions use variable amount of time. This section will contain
%information about how we normalize and compare energy consumption of
%these instructions. It will be difficult to compare single cycle instructions
%to the multi cycle ones, as the single cycle instructions is often the ones
%utilizing more than one ALU at a time. Also, the multi-cycle ones will most
%likely pipeline up very differently than the single cycle ones. We will try to
%draw a conclusion about the results, but it is important to note the differences
%in the execution path of these two categories.

\subsection{Instruction Level Energy Efficiency}

We distinguish between single-cycle instructions and multi-cycle instructions
because they behave differently in and around the execution pipelines.
Instructions using only one cycle are fairly easy to reason about as there is no
need to normalize energy consumption with respect to the cycle count (i.e.
time). However, it is important to also recognize CPU capabilities such as dual
issuing which are present on the processor: most single-cycle ALU instructions
execute pairwise in parallel -- one in each ALU -- giving a peak performance of
two instructions per clock cycle. Multi-cycle instructions needs to be carefully
considered. Typically, multi-cycle instructions divide work which can be done in
a subset of the available ALUs (e.g. one) over several cycles, and can therefore
introduce bottlenecks in the execution path. This again makes the processor do
less, lowering the average current drain. For all these reasons, we partition the
measured data in two data sets; one for single-cycle instructions and one for
multi-cycle instructions.

Figure \ref{fig:singlecycle}, \ref{fig:multicycle}, \ref{fig:allmul} and
\ref{fig:cond} displays our results from measuring current drain for each
instruction. The instructions is sorted in increasing order by Ampere cycles.
Green bars represents single-cycle instructions, light blue are two-cycle
instructions, while dark blue represents three-cycle instructions.  The red bar
to the left on each graph shows the baseline for current measurement.  The
baseline is an alias for the least power-consuming instruction we could find,
which is the \texttt{setend}-instruction. This instruction sets the endianness
for all memory operations to either big or little endian \cite{armcompilerref},
and has a current drain of only $161.3mA$ when executed repeatedly. This is
expected because it would force pipelines to be empty most of the time.

During measurements, $V_{core}$ was kept stable at $1.3V\pm50mV$, well within
the specifications of the processor. The pipelines were kept as full as
possible, avoiding hazards and instruction loading. This implies that
instructions utilizing large parts of the processor will most likely be more
energy consuming than those using only few components. This statement is
supported by the fact that the \texttt{setend}-instruction has little pipeline
activity at the same time as it has a low continuous current drain.

\subsection{Single-Cycle Instructions}
On our target CPU, 70 of the 115 tested instructions\footnote{119 including
conditionals} use a single cycle, while the remaining 45 uses 2 or 3.  Nearly
half of the instructions are multiply instructions, so these will be discussed
separately. \autoref{fig:singlecycle} displays a comparison of the 50
non-multiply single-cycle instructions.

\begin{figure}
    \centering
    \includegraphics[width=0.4\textwidth]{figures/graph_01_base_cond-0c6}
    \caption{Energy profile showing conditional execution.}
    \label{fig:cond}
\end{figure}

The results in \autoref{fig:singlecycle} shows that the ordinary single-cycle
instructions do not differ very much. An interesting result is how instructions
bearing the \texttt{s}-flag seems to have a lower consumption than their
non-\texttt{s} companion. These instructions updates status flags and will
likely force in-order execution. According to the performance counters in
\autoref{tab:perf_nonmul} there is reason to believe that the processor has to
stall one cycle between each issue. From our results, it seems that this saves
energy. However, the instructions needs longer time to complete, which is indeed
less energy efficient.

The results from the conditional-executed instructions are also subject
to forced in-order execution. We can see from \autoref{fig:cond} how variations
of \texttt{add} compares. In the figured test, \texttt{addne} is
committed every time, while \texttt{addeq} never has its results committed. It
is interesting to see that even though \texttt{addeq} is never committed, it
uses almost as much power as the other \texttt{add}s. By looking at
\autoref{tab:perf_nonmul} we see that \texttt{addeq} and \texttt{addne}
introduces a lot of both \emph{No disp.} and \emph{Issue Empty}. We do not know
exactly why, but it is reasonable to believe that one must assure that the previous
instruction did not alter the status flags before the results are committed or
discarded. In an in-order single-issue processor, conditional execution provides
a framework to avoid unnecessary jumps, while in an out-of-order core,
conditional execution is most likely much harder to implement.  Also note that
this test is very synthetic and the ISA is likely to be unoptimized for such
activity. In a real world workload, it is possible that the required
synchronization is hidden.

Further, \autoref{fig:singlecycle} shows that the \texttt{nop}-instruction has a
rather low power consumption. This is a bit misleading, as the
\texttt{nop}-instruction assembles to \texttt{mov r0, r0}, having both
read-after-write and write-after-write hazard on itself. This makes the
\texttt{nop}-instruction serialize itself, and it is hard to fill the pipeline
with this instruction.  Knowing this, it makes sense that \texttt{nop} works in
this way, as it is often used to fill out clock cycles with non-destructive
work. It would not make sense to optimize the \texttt{nop} instruction, as it
then would fail to complete it's goal as a space-and-time filler.

\begin{figure}
    \centering
    \includegraphics[width=0.48\textwidth]{figures/graph_023_base_quad_saturate_extend-0c6}
    \caption{Energy profile of multi-cycle instructions, excluding multiply.}
    \label{fig:multicycle}
\end{figure}

Generally, when accounting for the number of cycles used by the different
instructions, we see that the least current demanding single cycle instructions
are \texttt{add} and \texttt{sub} at $216.2mA$, while \texttt{rev} and
\texttt{sel} consumes slightly more with a drain of $225.4mA$. The measurements
have a standard deviation of $4mA$ and $3.7mA$, respectively. Overall, the
standard deviation ranges from $2mA$ to $7mA$, which we consider to be more than
good enough.

\subsection{Multi-Cycle Instructions}

%See \autoref{fig:multicycle} and \autoref{fig:allmul}
%\begin{itemize}
%    \item The difference between instructions are much larger in the multi-cycle
%        graphs.
%    \item Talk about mul, and why it might differ that much (especially 2 and 3 cycle ones)
%\end{itemize}

45 of the instructions that was compared used 2 or 3 cycles to complete their
results. 18 of these instructions are non-multiply. Non-multiply
instruction power measurements are displayed in \autoref{fig:multicycle}. A
selection of the performance counter results are shown in
\autoref{tab:perf_nonmul}. We see that the unsigned extend
instructions(\texttt{ux*}) are
slightly cheaper than signed extend (\texttt{sx*}). This might indicate that
some hardware is left idle when not needing sign extension.  The instructions
are normalized according to their stated cycle count in the table B-5 in
\cite{armtech}.

\begin{figure*}
    \centering
    \includegraphics[width=\textwidth]{figures/graph_0123_base_mul-0c6}
    \caption{Energy profile of multiply instructions.}
    \label{fig:allmul}
\end{figure*}

\begin{table}
    \centering
    \begin{tabular}{|p{0.7cm}|R{0.8cm}R{0.6cm}R{0.85cm}R{0.6cm}R{0.45cm}R{0.75cm}R{0.72cm}|}
       \rowcolor{gray!50}
        \hline
        \centering
        Instr. &
        \centering
        Cycles &
        \centering
        Main Ex. &
        \centering
        Second Ex. &
        \centering
        Pred.&
        \centering
        Mis pred.&
        \centering
        No disp. &
        \begin{centering}
        Issue Empty
        \end{centering}
        \\
        \hline
        mla&6608&6530&4895&251&2&76&76\\
mlas&15639&6529&12548&251&2&8857&73\\
muls&15617&6526&252&251&2&12100&61\\
smlabb&3604&3518&3265&251&2&83&83\\
smlabt&3604&3518&3265&251&2&83&83\\
smlad&3604&3518&3265&251&2&83&83\\
smladx&3604&3518&3265&251&2&83&83\\
smlal&7106&7028&5020&251&2&575&76\\
smlalbb&7102&7021&5019&251&2&3582&76\\
smlalbt&7102&7021&5019&251&2&3582&76\\
smlald&7102&7021&5019&251&2&3582&76\\
smlaldx&7102&7021&5019&251&2&3582&76\\
smlals&15888&6529&12548&251&2&9106&73\\
smlaltb&7102&7021&5019&251&2&3582&76\\
smlaltt&7102&7021&5019&251&2&3582&76\\
smlatb&3604&3518&3265&251&2&83&83\\
smlatt&3604&3518&3265&251&2&83&83\\
smlawb&3604&3518&3265&251&2&83&83\\
smlawt&3604&3518&3265&251&2&83&83\\
smlsd&3604&3518&3265&251&2&83&83\\
smlsdx&3604&3518&3265&251&2&83&83\\
smlsld&7102&7021&5019&251&2&3582&76\\
smlsldx&7103&7021&5019&251&2&3583&77\\
smmla&6608&6530&4895&251&2&76&76\\
smmlar&6608&6530&4895&251&2&76&76\\
smmls&6608&6530&4895&251&2&76&76\\
smmlsr&6608&6530&4895&251&2&76&76\\
smmul&6602&6525&252&251&2&3336&76\\
smmulr&6602&6525&252&251&2&3336&76\\
smuad&3602&3266&252&251&2&334&334\\
smuadx&3602&3266&252&251&2&334&334\\
smulbb&3353&3267&252&251&2&84&84\\
smulbt&3353&3267&252&251&2&84&84\\
smull&6858&6779&6774&251&2&327&77\\
smulls&15637&6528&15558&251&2&8856&72\\
smultb&3353&3267&252&251&2&84&84\\
smultt&3353&3267&252&251&2&84&84\\
smulwb&3353&3267&252&251&2&84&84\\
smulwt&3353&3267&252&251&2&84&84\\
smusd&3602&3266&252&251&2&334&334\\
smusdx&3602&3266&252&251&2&334&334\\
umaal&7106&7028&5020&251&2&575&76\\
umlal&7106&7028&5020&251&2&575&76\\
umlals&15888&6529&12548&251&2&9106&73\\
umull&6858&6779&6774&251&2&327&77\\
umulls&15637&6528&15558&251&2&8856&72\\

        \hline
    \end{tabular}
    \caption{Performance counter data from 252 iterations of all tested multiply
    instructions.}
    \label{tab:perf_mul}
\end{table}


\subsection{Multiply}
The ARM Cortex-A9 contains a single multiply pipeline, but has two general
ALUs\autoref{fig:pipeline}. The multiply instructions are queued up waiting to
execute through the same pipeline. This implies that multiply instructions would
have a lower continuous power drain because it does less useful work and will
seemingly use less energy compared to instructions utilizing both pipelines at
its full potential. We have not compensated for this matter other than
multiplying the power drain with the number of cycles used to finish one
multiply instruction. It is unknown how the different multiply instructions
utilize the pipeline(s). As we can see from \autoref{tab:perf_mul}, there is
reason to believe that at least some of the multiply-accumulate instructions
utilize both pipelines\cite{ramangcc}. This means that some instructions are
able to utilize more hardware while still queuing up through the
multiply-enabled main pipeline.

By looking at \autoref{fig:allmul} we see that the single-cycle multiply
instructions are quite similar, but those using two or three cycles are more
interesting. We do not know why the results are as stated, as most of the
internal architecture are not available for the public. According to
Table B-5 in \cite{armtech}, some multiply instructions uses more time than others
before the result is available.

From the performance counters in \autoref{tab:perf_mul}, we see that
instructions are treated differently by the architecture. We have not considered
all the tested instructions in detail, but it is evident to us that there is a
strong negative correlation between performance counters (\emph{No Disp.} and
\emph{Issue Empty}) and processor power drain. The results in
\autoref{fig:allmul} shows power drain in Amperes multiplied by the cycle
counts. The values are not normalized according to the performance counter
values.

\subsection{Evaluation}
\begin{figure}
    \centering
    \includegraphics[width=0.48\textwidth]{figures/heat}
    \caption{Changes in heat and energy consumption for \texttt{add} at
    different runs together with heatsink and ambient temperature}
    \label{fig:heat}
\end{figure}

Each instruction was measured 41 times. We found small variations in power
consumption between testbench runs, but all results shows the same trend. As
stated in \autoref{sec:temperature}, the power consumption is not easily pushed
by temperature. \autoref{fig:heat} shows how the change in power consumption of
the instruction \texttt{add} over different runs combined with the ambient
temperature and the heatsink temperature. According to these results, we assume
that the change in power consumption was not due to heat. We did not log
temperature for all test runs, but assume that the results from
\autoref{fig:heat} holds, and that this small change in temperature is at least
not solely responsible for variations in the current drain measurements.
