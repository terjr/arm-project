\section{Instruction level energy consumption} \todo[inline]{change title to
Rationale?}

\begin{itemize} \item Describe why the procedure we present gives a reasonable
    \end{itemize}

% General approach here? Implementation in methodology? Order of apperance?

The need for energy efficent processors is increasing, and in order to achieve
better understanding of how architectural decisions, we propose a method to
measure energy efficency on the instruction level of a processor.

Our scheme relies on fairly accurate core voltage and power measurements, thus
we have to modify the processor in such a way that core power can be feed
externally.  We run a customised instruction loop, one for each applicable
instruction of the chosen ISA, and measure its power drain.

As each instruction cannot be measured alone, we create an assembly loop that
contains the instruction under test. This loop runs long enough to be measurable
by ordinary laboratory equipment.

It is a common philosophy that a RISC should consume one clock cycle pr.
instruction\cite{sivarama}. Even so, this is not the most common case any more.
Dual pipelines and dual issue and parallell general ALUs are entering RISC
processors, allowing the processor to retire more than one instruction pr. clock
cycle. At the same time, processors that support multiply, divide and floating
point often use different amount of clock cycles to perform these operations. It
might even be so that multiply, divide and floating point takes different amount
of cycles according to their surrounding instructions.

Since some instructions takes a different amount of time, the power drain has to be
normalised using statistics gained from performance counters. Applying this
normalization, we can convert point-in-time energy consumption in terms of
wattage to energy per instruction in terms of joules.

As we need data from the performance counters, the measured instruction loop has
to be conditional. Thus the list of equal instructions has to be
ended by a conditional branch, but their impact can be mitigated using
statistics and by knowing how the processor works. The effect of the 


