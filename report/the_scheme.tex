\section{Instruction level energy consumption}
\todo[inline]{change title to Rationale?}

\begin{itemize}
    \item Describe why the procedure we present gives a reasonable
\end{itemize}

% General approach here? Implementation in methodology? Order of apperance?

The need for energy efficent processors is increasing, and in order to achieve
better understanding of how architectural decisions, we propose a method to
measure energy efficency on the instruction level of a processor.

Our scheme relies on fairly accurate measurements of each instructions, in terms
of energy. Since each instruction cannot be measured alone with normaly
available instruments, we set up an assembly code loop that fits in the
processors instruction queue and makes it large enough to allow power
consumption measurements. The list of equal instructions has to be ended by
a conditional branch, but their impact can be mitigated using
statistics and by knowing how the processor works.

The reason for why we need a conditional branch is because we disable
interrupts, but stil want the test to end after a reasonable amount of time.
Also, we want to read out the performance counters after each test in order to
do instruction normalisation.


